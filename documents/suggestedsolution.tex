\documentclass[10pt,a4paper]{article}
\usepackage[utf8]{inputenc}
\usepackage{amsmath}
\usepackage{amsfonts}
\usepackage{mathtools}
\usepackage{amssymb}
\usepackage{graphicx}
\usepackage{subcaption}
\usepackage{dsfont}
\usepackage[strict]{changepage}
\usepackage{float}
\usepackage{parskip}
\makeatletter
\newcommand*{\transpose}{%
  {\mathpalette\@transpose{}}%
}
\newcommand*{\@transpose}[2]{%
  % #1: math style
  % #2: unused
  \raisebox{\depth}{$\m@th#1\intercal$}%
}

\author{Alexander Israelsson \texttt{israelsson.alexander@gmail.com} \and
Adam Jalkemo \texttt{adam@jalkemo.se} \and
Emil Westenius \texttt{emil@westenius.se} \and
Jonathan Andersson \texttt{mat11ja1@student.lu.se}}
\title{Adaptive Friction Compensation}
\begin{document}
\maketitle

\section{Project background}
In this project a Furuta pendulum process will be used. The pendulum will be stabilized in the inverted position using a swing up controller and a top controller. Further the friction will be considered and estimated using an adaptive method.

\section{Controller}
For the swing up a Lyapunov based controller will be used. In the top position a LQG controller will be used. The controllers will be implemented in Java.
Models for Coulomb and viscous friction and additional models for asymmetric friction will be considered. 

\section{Time plan}
Start week 14, done week 15

\section{Responsibilities}
Theory
	What estimators to use
	What friction model
Friction controller in MATLAB
Controller in Java
	Friction
	Swingup
	Top controller
Program skeleton in java
GUI
Report
Presentation




\end{document}

